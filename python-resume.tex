% Created 2015-05-21 Thu 13:17
\documentclass[11pt]{article}
\usepackage[utf8]{inputenc}
\usepackage[T1]{fontenc}
\usepackage{fixltx2e}
\usepackage{graphicx}
\usepackage{longtable}
\usepackage{float}
\usepackage{wrapfig}
\usepackage{rotating}
\usepackage[normalem]{ulem}
\usepackage{amsmath}
\usepackage{textcomp}
\usepackage{marvosym}
\usepackage{wasysym}
\usepackage{amssymb}
\usepackage{capt-of}
\usepackage{hyperref}
\tolerance=1000
\usepackage[utf8]{inputenc}
\usepackage[scaled]{helvet}
\renewcommand*\familydefault{\sfdefault}
\author{Oleg Sivokon}
\date{\textit{<2015-05-19 Tue>}}
\title{Resume}
\hypersetup{
 pdfauthor={Oleg Sivokon},
 pdftitle={Resume},
 pdfkeywords={Resume, job, employment, cv},
 pdfsubject={My resume},
 pdfcreator={Emacs 25.0.50.1 (Org mode 8.3beta)}, 
 pdflang={English}}
\begin{document}

\maketitle

\section{Contact Information}
\label{sec:orgheadline1}
\begin{center}
\begin{tabular}{ll}
Name: & Oleg Sivokon\\
Birth date: & 23, 06, 1978\\
Cell phone: & +972 (0) 544-563314\\
Email: & \href{mailto:olegsivokon@gmail.com}{olegsivokon@gmail.com}\\
Address: & Florentin 50, Tel Aviv, Israel\\
\end{tabular}
\end{center}

\emph{You can find an up-to-date electronic copy of this document at}

\url{https://github.com/wvxvw/resume}

\clearpage

\section{Foreword}
\label{sec:orgheadline2}
After sending my complete resume couple dozens times I've came to conclusion
that if I mention all programming-related experience I had so far, this
confuses human resources employees, therefore this version of my resume
will focus on my experience related to Python and will deliberately leave
out the rest.

I will, however, put a loundry list of technologies I worked with here in
order to optimize the search.

\begin{description}
\item[{GNU/Linux}] is the operating system I use at work and at home.  Most of
the time it's RHEL-like distros, but I'm also familiar with Debian-style
infrastructure.  I know enough to perform a system administrator function,
however I'm not actively seeking that role.
\item[{Web stack}] which includes HTML, JavaScript and CSS.  I've used all of
these on more than one occasion.  In fact, my programming career began
with another ECMAScript dialect - ActionScript.  I'm familiar with
many JavaScript implementations, including NodeJS, Rhino and even
Qt Script.  I'm familiar with NodeJS because it is used with tools
like \texttt{jslint}, \texttt{tern} and \texttt{Grunt} for JavaScript development tasks.
I'm familiar with Rhino because it is used in testing in Selenium
and because it is available in Ant scripts, while it doesn't require
additional libraries, which makes it a choice of embedded programming
language in it.  I was considering Qt Script for one of my pet projects,
which, unfortunately, never left the ground.
If I need to throw in some names, then, of course, I used JQuery, 
I also used Backbone before everyone was so excited about Angluar,
I also used Prototype before everyone was so excited about Backbone.
Of course I know and used Underscore, various JQuery plugins, Mootools,
and a lot of other species from the JavaScrip zoo.
Eventually, I also touched PHP, JSP and even ASP (classic!) both the
Basic and the JScript varieties, but washing the smell off my hands
was no easy taks.
\end{description}

\section{Work History}
\label{sec:orgheadline6}

\subsection{PowToon Ltd.}
\label{sec:orgheadline3}
\begin{itemize}
\item \textbf{Position:} \emph{Programmer}
\item \textbf{Term:} \emph{January 2014 – May 2015}
\end{itemize}

When the company hired me, they didn't have any automated builds and no tests
for their product.  Even though it wasn't my direct responsibility, I
volunteered to write build scripts, which included using Fabric, Ant and
Gradle.  I also volunteered to set up Jenkins continuous integration server
which I later managed until I quit.

I also tried to organize testing, however, I had only achieved partial success
here.  I wrote a minimal HTTP server based on \texttt{SimpleHTTPServer} with HTML
interface for developers, and (later) testers, where they could submit bug
reports, which would be automatically converted into test cases.  Regardless
of the system being very simple, other programmers avoided using it.

During my time at Powtoon I also conducted several experiments in statistical
analysis (albeit very simplistic) of the data collected by the company.
In particular, using \texttt{NumPy} and \texttt{SciPy} packages I wrote a simple \emph{k-means}
algorithm which would try to cluster the textual data extracted from the
presentations prepared by the users based on the presentation's subject.

Powtoon's server-side code is written using Django framework, obviously,
I had to use it too.

\subsection{TransGaming Inc.}
\label{sec:orgheadline4}
\begin{itemize}
\item \textbf{Position:} \emph{Programmer}
\item \textbf{Term:} \emph{April 2013 – 2013 (less than a year)}
\end{itemize}

A very cumbersome part of my job here was the testing.  The company
specializes in games for the so-called ``smart-TV''.  This implies working
with proprietary products which make it difficult to test integration due to
restrictions wrt viewing the source code and communication protocols.

In order to automate the testing, I used Selenium server, which I extended
with my code intended to automate deployment into television set or an
emulator.  My job, however, had an interesting side to it too: since my
primary objective was to write a game, in order to test the game, I wrote
several game bots, with different skill level.

\subsection{Rounds}
\label{sec:orgheadline5}
\begin{itemize}
\item \textbf{Position:} \emph{Programmer}
\item \textbf{Term:} \emph{December 2010 – August 2011 (9 months)}
\end{itemize}

Before the company hired me, the company never conducted any systematic
testing of their product.  Even though testing was not on the list of my
direct responsibilities, I volunteered to write a minimal testing server
which would replicate the API of the company's main server and I wrote a chat
client (the company's product is a video chat), which would enact prepared
scenarios in order to search for misalignment in the chat protocol
interpretation on the client and server sides.

\section{Courses}
\label{sec:orgheadline7}
I took several online courses from Coursera, Udacity and edX, which required
me to complete programming assignments in Python.

I took a complete series of courses in statistics at edX (three courses),
the data scientist toolbox course at Coursera (six short courses), machine
learning at Udacity, introduction to algorithms (two courses) at Coursera.

I also volunteered for SCons project, writing an ActionScript plugin for it.
(SCons is an extensible build system written in Python).

Two years ago I interviewed for a position of Python programmer at Walla.
They requested that I write a small test project, which I did.  The project,
although somewhat outdated, is still available for viewing:
\url{https://github.com/wvxvw/intj}.

I am currently studying in Open University of Israel, mathematics faculty,
about to finish my first year.

\section{Hobbies}
\label{sec:orgheadline8}
\begin{itemize}
\item NLP
\item Computational linguistics
\item Moral philosophy
\item General AI
\item Graphs
\end{itemize}
\end{document}