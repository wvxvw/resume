% http://www.ctan.org/tex-archive/macros/latex/contrib/memoir
\documentclass[11pt,a4paper,oneside]{memoir}
\usepackage[english]{babel}
\usepackage[utf8]{inputenc}
% http://www.ctan.org/tex-archive/macros/latex/contrib/enumitem
\usepackage{enumitem}
\usepackage{epigraph}
% Without this, the title page will not compile correctly.
\newlength{\drop}
% To avoid using drop, see: http://wiki.lyx.org/LyX/UsingMemoirInLyX

% See p92 of the Memoir manual.
\chapterstyle{demo2}

\pagestyle{myheadings}

\setlength{\parindent}{0pt}

% Set the title of the contents page.
\renewcommand{\printtoctitle}[1]{\centering\Large\bfseries Acts}
% Remove the title from the contents page entirely.
% \renewcommand{\printtoctitle}[1]

% Remove page numbers until told otherwise.
\pagenumbering{gobble}

% Various title pages may be used with the memoir package.  The one
% below is from ``Some Examples of Title Pages'' (Peter Wilson) at
% http://www.ctan.org/tex-archive/info/latex-samples/TitlePages.

% Set up the title page.
% Gentle Madness title page style
\newcommand*{\titleGM}{\begingroup
  \drop = 0.1\textheight
  \vspace*{\baselineskip}
  \vfill
  \hbox{%
    \hspace*{0.2\textwidth}%
    \rule{1pt}{\textheight}
    \hspace*{0.05\textwidth}%
    \parbox[b]{0.75\textwidth}{
      \vbox{%
        % Main title of the play
        \vspace{\drop}{\noindent\HUGE\bfseries The Resume}\\
        % \vspace{\drop}{\noindent\HUGE\bfseries
        % Title of the play \\[0.5\baselineskip] over two lines}\\
        % Subtitle of the play
        [2\baselineskip]{\huge\itshape A Farce}\\
        % [2\baselineskip]{\Large\itshape Subtitle of the play \\
        % [0.5\baselineskip] over two lines}\\
        % Author of the play
        [4\baselineskip]{\Large Oleg Yevgenievich Sivokon}
        \par\vspace{0.5\textheight}
        % [4\baselineskip]{\Large First Author \\
        % [0.5\baselineskip] Second Author \\
        % [0.5\baselineskip] Third Author \\}\par\vspace{0.5\textheight}
        % Publisher and year of publication
        {\noindent \textbf{Oleg Y. Sivokon}\\
          [0.5\baselineskip] \textbf{2013}\\
          [0.5\baselineskip] \itshape Born: 23.06.1978\\
          [0.5\baselineskip] \itshape Cell phone: +972 (0)544563314\\
          [0.5\baselineskip] \itshape Electronic mail: olegsivokon@gmail.com \\
          [0.5\baselineskip] \itshape Address: Florentin 50, Tel Aviv, Israel}\\
        [\baselineskip]
      }% end of vbox
    }% end of parbox
  }% end of hbox
  \vfill
  \null
  \endgroup}

\begin{document}

\epigraph{No, I'm not interested in developing a powerful brain. All
  I'm after is just a mediocre brain, something like the President of
  the American Telephone and Telegraph Company.}{Alan Turing}

\clearpage
% Print out the title page.
\titleGM

% Start numbering pages with Roman numerals (for the front matter).
\pagenumbering{roman}

% Print out the contents page, listing the acts of the play.
% You will need to run pdflatex twice before the page numbers show up.
\tableofcontents*
\clearpage

% Print out the characters page, listing the dramatis personae
% The starred form of \chapter prevents a chapter number
% (eg ``One'', ``Two'') being printed before each chapter title
% (eg ``Characters'', ``Act 1'').
\chapter*{CHARACTERS}
% Centre the list of characters.
% Comment out this line and \end{}center if centring is not desired.
\begin{center}
  \textbf{Oleg}, the interviewed, I\\
  \textbf{Ronen}, the interviewer, a chief programmer of a gambling company\\
  \textbf{Shira}, the interviewer, a human resources expert\\
  \vskip 1cm

  \textbf{Scene}: Oleg's room.\\
  \textbf{Time of action}: Early morning.
\end{center}

% Print out a page with any additional authorial comments,
% notes on staging, or whatever.
\chapter*{THE PLOY}

% Set up a description list to hold the paragraphs.
% Increase the space between the list items, and set
% the left margin to 0.20cm
\begin{description}[itemsep=1ex,leftmargin=0.20cm]

  % Precede each paragraph with an empty \item[].
\item[] The events you are about to witness took place when the
  narrator himself experienced them. This play is a true and honest
  report from the field of head hunting in programming industry.

\item[] The author admits to visiting of at least thirty job
  interviews in the last fiscal year only, thus collecting the
  experience required to compile this play.

\item[] Readers are expected to be familiar with the job interviewing
  process. Basic understanding of philosophy, set theory, category
  theory, algebra, theoretical computer science, artificial
  intelligence, theory of mind, folk psychology, computational
  linguistics and formal languages may be instrumental.

\end{description}

\clearpage
% Start numbering pages with Arabic numerals (for the text of the play).
\pagenumbering{arabic}

% Generate a running header with the title of the play.
\markright{\textsc{The Resume}}

%%%%%%%%%%%%%%%% 
\chapter*{ACT 1}
%%%%%%%%%%%%%%%% 
% The \chapter* will prevent the the chapters (Acts) being listed in
% the table of contents, so we need to add them manually.
\addcontentsline{toc}{chapter}{Act 1}

% The starred form of \section prevents a section number
% (eg ``1.1'', ``2.3'') being printed before each section
% title (eg ``Scene 1'', ``Scene 2'').
\section*{\hfill\textit{scene 1}}
% \section*{\hfill\textit{SCENE 1}}
% Use this line instead if you want the Scene 1 heading shifted to
% the right edge of the page.

% Set up a description list to hold the dialogue of the scene.
% Increase the space between the list items, and set the left
% margin to 1cm.
\begin{description}[itemsep=1ex,leftmargin=1cm]

  % Where the scene or act begins with stage directions:
\item[] \hfill \\
  \textit{Semi-lit untidy small room. Oleg sists in front of his
    computer peering at the mail accumulated during the night. The phone
    rings.}

  % Where the scene or act begins with dialogue, and no stage directions:
  % \item[] \hfill

  % Wrap each character's name in \item[].
  % Wrap in-dialogue directions in \textit{}
\item[Oleg] \textit{(faking enthusiasm)} Hello!

\item[Oleg] Yes, it is still relevant.

\item[Oleg] Yes, it is convenient to talk now.

\item[Oleg] \textit{(after a long pause)} You see, it's a loaded
  question... Actually, you know what, I wouldn't mind answering it
  in-depth, but I'm afraid you will find it teditious to follow.
  I've been told I can sound condescending when I need to explain
  myself. I have to confess that I am, some times, when I...

\item[Oleg] Yes, yes, I see.

\item[Oleg] See-es-es three? Yes.

\item[Oleg] Ech-tee-em-el five? Why, of course...

\item[Oleg] Since... since... well, you see it's hard to answer. The
  format itself didn't change that much since its very early version.
  So, I guess...

\item[Oleg] OK, never mind. Make it f-f-five, three years. Yes, three.

\item[Oleg] \textit{(after yet another pause)} I'm sorry, I don't
  think I ever mentioned that. Come again please?

\item[Oleg] El-eye-es-pee? A language? A programming language that I
  know? Wait, that can't be right... Oh, I see now. Well, scratch
  that. It's a technical thing, you don't need to know.

\item[Oleg] No, no, never mind. No one would search for a Lisp
  programmer. It's like in that Dillbert comic...

\item[Oleg] Dillbert, who's Dillbert? Well, just never mind it then.
  It's not important really.

\item[Oleg] Yes, thank you, have a nice day! I'll let you know if
  they contact me.

  % Close the description list at the end of the scene.
\end{description}
\vskip 1cm  % Put a bit of space between this and the next scene heading.

\section*{\hfill\textit{scene 2}}
\begin{description}[itemsep=1ex,leftmargin=1cm]

  % Where the scene or act begins with dialogue, and no stage directions:
\item[] \hfill \\
  \textit{Same room. Oleg sists in front of his computer peering at
    the mail accumulated during the day. The phone rings.}

\item[Oleg] \textit{(faking enthusiasm)} Hello!

\item[Oleg] Yes, the HR people warned me.

\item[Oleg] Do you have my e-mail?

\item[Oleg] Great.

\item[Oleg] Yes, any time really, I'm not doing anything important
  this week.

\item[Oleg] No, I don't do anything imporatnt the next week either.

\item[Oleg] Yes, I'll send you a confirmation in a moment.

\item[Oleg] So Tuesday at eleven? Great.

\item[Oleg] Me too.

\end{description}
\vskip 1cm


%%%%%%%%%%%%%%%% 
\chapter*{ACT 2}
%%%%%%%%%%%%%%%% 
\addcontentsline{toc}{chapter}{Act 2}

% The \hfill will shift the scene heading to the right edge of the page.
\section*{\hfill\textit{scene 1}}
\begin{description}[itemsep=1ex,leftmargin=1cm]
  \setlength{\parskip}{5pt}

\item[] \hfill \\
  \textit{At Ronen's office.}

\item[Ronen] \textit{(faking a smile)} I've read your resume, and may
  I say it is strange! Why on Earth would you make it a dialogue?

\item[Oleg] \textit{(smiling mordantly)} Oh really, have you read it?
  Most people in your position won't make it past the front page.

\item[Ronen] Yes, I only took a glimps of the first part, but now that
  you mention it, let me...

\item[Oleg] I wouldn't do it if I were you.

\item[Ronen] ???

\item[Oleg] Well, you see, it is a paradox. No. It is actually a gamble.
  Do you like to gamble?

\item[Ronen] \textit{(still faking a smile)} Seriously?

\item[Oleg] You have a one in two chance of never finishing to read the
  resume, if you consider the following proposition:\\
  Say, and the resume is fake, and it doesn't describe our meeting as
  truthfully and honestly as it had declared already--then you loose
  nothing. Perhaps, you might have a good laugh and that would be all
  of it.\\
  But I dare you to consider a morbid possibility of the resume delivering
  on its promise! What if you were to read about how you would read
  the resume, and then you would read about how you reflected upon what
  you read and then you would read about how you reflected upon your
  reflection--do you see the problem?

\item[Ronen] \textit{(boldly rummages through the pile of papers on
    his desk)} Hmmm... that would be... recursive?

\item[Oleg] As a matter of fact--no. It would be infinite. Recusions
  have a termination condition, however in general, you can't prove
  that an operation is recursive, because you can't guarantee that
  the termination condition will be ever reached.\\
  Remeber Alan Turing?--Yes, like that.

\item[Ronen] \textit{(never finding the resume)} OK. To hell with it.
  So would you mind telling me about yourself?

\item[Oleg] Should I start from the most recent and go back to the
  beginning, or should I keep the chronologic order?

\item[Ronen] Whatever, it doesn't matter.

\item[Oleg] Oh, it sure does. Remember the previous problem? It has
  been noticed by Korzybski that in order to exchange the knowledge
  we must find the relevant patches of the territory in both of our
  brains, such as to ensure understanding.\\
  And let me tell you that if we start from today and go all the
  way back into the past, then you might find it easier to understand
  who are you talking to, but experiencing events in reverse order
  may not be the easiest way to proceed.\\
  On the other hand, watching the evnets unfold in the same order
  they happened is the easiest way, but you will have to start with
  a very wrong image of me and you will have to evolve it with the
  story.\\
  Also, my intuition hints me that you will be anxcious to quit
  listenning. Thus, it is in my best interest to go backwards in
  time--whilst it's the opposite for you.

\item[Ronen] OK, reverse. No, whatever. \textit{(looks at his iPhone)}

\item[Oleg] I think I have just the story for you!

\item[Ronen] Yes, that's better.

\item[Oleg] I interviewed for a similar position at the start-up
  company. The interviewer, who was, coincidentally, called Ronen,
  had me explain him object-oriented design. He was unconvinced by
  me referencing Ostersky as an authority on \textit{vObject calculus},
  instead he opted for a thought experiment.\\
  ``Tell me, Oleg, how would you program an elevator?''--inquired he.\\
  ``I should warn you, I don't see how objects come into play here,
  but it would be my pleasure to explain it to you, but you need to
  be more exact: please explain, how do you want the elevator to operate?''\\
  \textit{Ronen was starting to get anxcious}\\
  ``What's the matter? It's just an elevator, like every other elevator!''\\
  ``Do you want an efficient or a fair elevator?''\\
  ``Is there a difference?''--Ronen was already about to loose his temper.\\
  ``Efficient elevators accomplish their task faster, but they may appear
  frustrating to the passengers.''\\
  ``Efficient.''\\
  ``Great. How many elevators are there in the building? Should they be
  aware of each other?''\\
  ``Just a generic elevator. It doesn't matter how many of them are in
  the building. I just need you to write the program in pseudocode''--Ronen
  was trying to get the answer he wanted, not realizing how flawed the
  question was.\\
  ``Then pseudocode be it!''--that said, I took a shit of paper and began
  to draw.\\
  ``So... let's define our elevator in terms of what it does: it can
  move up, let's call it a `U'; down shall be `D'; `L' for load and
  `N' for noop''\\
  ``We shall measure the efficiency of the elevator in the number of passengers
  delivered over time--that sounds fair, does it?''\\
  ``Yes, just move on.''--muttered Ronen.\\
  ``Here's our very simple program then:''\\
\begin{verbatim}
LUUUUUUUUUULDDDDDLDDL
\end{verbatim}
  ``Which reads as follows: pick up a passenger, move ten floors up, unload
  the passenger, move five floors down, pick another passenger, move yet
  two more floors down and unload the passenger. The total score of
  $21/2=10.5$--This elevator is fair, but inefficient.''\\
  ``What's inefficient about it?''--grunted Ronen.\\
  ``Oh, wait till you see another program!''\\
\begin{verbatim}
LUUUUULDDLUUUUUL
\end{verbatim}
  ``Do you see how it's efficiency is $16/2=8$? Ever wondered why no real
  elevators operate like that?''\\
  Ronen was clearly disappointed by the turn of events. He didn't want to
  count the ``U''s and ``D''s in my program, instead he told me:\\
  ``I don't care if it's efficient or not, it is not what I wanted to know.
  If I wanted an efficient elevator, I'd pay a professor to write an algorithm
  to do it.''\\
  ``Well, if such is the case, I'd rather work for the professor you hire,
  then for you. I was under impression I'm intervieweing for programmer's
  position.''\\
  We shook hands and never spoke again.

\item[Ronen] And the moral of the story?.. is that you can't cooperate?

\item[Oleg] The way you pose the question reminded me of just another
  story...

\item[Ronen] Please, no more stories.

\item[Oleg] Allright, I'll go right for the moral then. Have you ever
  heard about fuzzy logic? Let me give you an example:\\
  \begin{adjustwidth}{24pt}{16pt}
  \textit{All John's children are bald.}\\
  \end{adjustwidth}
  If you choose any first order logic, you are not be allowed to infer
  any of the following, however it might feel natural to you to do so:\\
  \begin{enumerate}
  \item John may not be three years old. \textit{John doesn't need to be
      a human.}
  \item John has children. \textit{In case John doesn't have children,
      the proposition above is still true.}
  \end{enumerate}
  Or, here is another example, proposed by Austin:\\
  \begin{adjustwidth}{24pt}{16pt}
    \textit{Suppose I order a cage from a carpenter. Suppose I tell the
      carpenter that the cage is for my bird. The carptenter makes a
      cage with the roof. If I then sue the carptenter, arguing that since
      my bird is a penguin he should not have the roof attached to the cage,
      thus he should not charge me for the extra work--I will certainly
      loose, because in the common practice an illogical, but a very likely
      condition is believed to be true.\\}
  \end{adjustwidth}
  To contrast humans, modern computers don't tolerate fuzzy logic.
  A program is believed to be flawless, when it covers all and every possible
  case of its usage. Programmers, thus, are required to be fluent in low-order
  logics in order to talk to computers.\\
  In the same way, you shouldn't extrapolate the example to include yourself
  unless you are confident that you would have behaved the same way
  \textit{the other} Oren did.

\item[Ronen] Are you a robot? \textit{(he tried not to look confused, but
    he was certanily working his way towards the end of the interview.)}

\item[Oleg] This is an interesting question. Of course I don't intend to
  take it literally, because in the very broad sense, not only I, but you,
  your cat and the flees in your cat's fur are robots...

\item[Ronen] My cat?

\item[Oleg] Sorry. \textit{I didn't mean it literally.} Hope you don't
  mind it if I continue?

\item[Ronen] Not at all.

\item[Oleg] There is a naive belief that robots are different from humans
  in that they don't mind repetitive tasks. This must owe to the fact
  that computers, those which operate robots, must be serial, whilst
  humans are massively parallel. The world is parallel too. So robots
  are, in a sense, out of this world, a kind of lunatics.\\
  There is, however, a thrilling contradiction in this observation!
  There are other outstanding traits non-robot humans share. I hope
  you will agree with this list:
  \begin{enumerate}
  \item Robots are unlikely to marry, to have a family or to even
    participate in any of the social interactions potentially
    leading to reproduction. Robots don't see the point of continuing
    their kind through replication.
  \item Robots are very unlikely to make mistakes. In fact, a mistake
    may upset the robot so much, it might burn out and explode whilst
    trying to cope with it. Humans, on the contrary, welcome mistakes,
    believing mistakes to be their true nature.
  \item Robots take no interest in sports, arts or non-traditional
    medicine, because they generally don't engage in mindless activites.
    To contrast that, humans would go to war for things like religion,
    art or to support their hometown football team.
  \item Robots don't value food. Actually, they abhore everything that
    could have them distracted while performing their sole task--since
    as we already mentioned, robots are single-threaded serial beings.
  \item Robots are never random. Randomness is like a virus to robots:
    it degenerates their brains, it destroys their predictable soft
    and deterministic picture of the world.
  \end{enumerate}
  In other words, you would agree with me that someone, who is a big
  sports fan, a married man with a lover or two on the side. A man,
  who might occassionaly sleep in, bring his dog to the office, or
  bring his home-cooked food to the lunch break looks like an exemplary
  human.\\
  Surprise: the same difference is reflected in language! Not only
  in programming languages, but, and especially so, in natural
  languages. In the realm of natural languages, it goes by the name
  of Grice's maxims. Or, rather... as flauting of these maxims.\\
  You see, humans exploit the faults in language semantics, creating
  simultaneously new meanings on pragmatic level.

\end{description}
\vskip 1cm

\section*{\hfill\textit{scene 2}}
\begin{description}[itemsep=1ex,leftmargin=1cm]
  \setlength{\parskip}{5pt}

\item[] \hfill \\
  \textit{The North Parade.}

\item[LUCY] So, I shall have another rival to add to my mistress's list, Captain Absolute.

\item[SIR LUCIUS] Hah! My little embassadress!

\item[LUCY] \textit{(speaking simply)} O gemini!  And I have been waiting for your worship here on the North.

\end{description}
\vskip 1cm


\end{document}
